\section{Background}

% \lettrine[nindent=0em,lines=3]{L}  

As the world increasingly comes online, one should expect a version of our political activities being performed on these platforms too. The affordances of the internet and networked platforms like Twitter, Reddit, and Facebook reduce the transaction costs of engagement at scale. By reducing the cost of content creation and downgrading the importance of traditional mediators of attention and political influence like the news media, platforms have enabled the amplification and shaping of erstwhile niche or marginalized narratives. These very affordances make political communication and activism a natural fit for actors on these platforms.

Political communication has always relied on tapping into pop culture to make its message more palatable to the public and likely voters. This has extended to digital politics too: political memes perform the role of satire, candidates maintain a presence across multiple platforms, which pollsters and campaigns rely on for fundraising and electoral analysis. But digital politics also means that a candidate or party's messaging on these platforms cannot be a controlled and mediated spectacle like it could have been in traditional media. Near-instant feedback and engagement, algorithmic amplification, and the sheer scale of platforms create an ecosystem where no campaign can reasonably hope to exert top-down control on narratives, messaging, and language. 

If President Obama's re-relection in 2012 was the first US presidential election that relied on social media for targeted messaging, Donald Trump's election win in 2016 marked the next stage of digital politics, driven by micro-targeted advertising, candidate tweets, data breaches, and disinformation campaigns. In addition to their messaging success, Trump's base proved to be extremely good at setting the agenda -- re-framing the controversy around Hillary Clinton's e-mails as a referendum on her lack of honesty, the overtly racist framing on issues like `Black Lives Matter', etc. Users of 4chan/8chan's r/pol/and reddit's \textit{r/The\_Donald} were able to command attention online, driving their narrative through memes and highly specific language familiar to digital natives. In light of all this, how does one track the changing context around issues in these communities? How does a user's linguistic behaviour change with greater engagement? How does one measure the embedded context around political entities on these forums?   

President Donald Trump’s rise has been associated with the rise to prominence of far-right communities, branded as the ‘alt-right’. The violent, ostensibly supremacist ‘Unite the Right’ rally in Charlottesville, Virginia was seen by many as a watershed moment that marked the rise of this subgroup \citep{atkinson_charlottesville_2018}. President Trump’s election campaign itself was marked by several racist and misogynistic statements, which were cheered on wildly and amplified by this subgroup. Heikkilä argues that by actively antagonizing both Democrats and classic conservatives -- through memes and online culture wars -- the alt-right pushed the conversation farther to the right, normalized the vilification of ‘political correctness’, and created a feedback loop that rewarded the campaign for its extreme positions \citep{heikkila_online_2017, nagle_kill_2017}. 

The tactics of the alt-right are heavily influenced by the 'ironic' meme culture of 4chan/8chan. The origins and inspirations for its more virulent side, which includes shitposting, bullying, targeted harrassment, doxxing (release of private information), can be traced to 'GamerGate'. ‘GamerGate’ refers to the targeted harassment of video game journalists  orchestrated through the hashtag '\#GamerGate'. Beginning in August 2014, it involved the large-scale coordinated harassment of specific women video game journalists who were critical of the portrayal of women in video games. Characterized as an effort to resist the imposition of politically correct norms on the art they love, the community lashed out at specific targets. Victims received death threats, rape threats, their addresses were posted online (doxxing). GamerGate was a vital moment in the crystallization of the alt-right as we know it \citep{blodgett_ghostbusters_2018}. The alt-right embraced the tactics of GamerGate, employing and co-opting the meme-first approach to discourse, along with a full-throated disavowal of feminist, politically-correct discourse. All of this was interspersed with 'ironic' jokes replete with racist, anti-Semitic, and misogynistic themes.

The importance of GamerGate in understanding the alt-right movement cannot be understated -- the incident tapped into a well of discontent in a population that wielded a lot of power in the online world, but felt marginalized. The misogynistic nature of the alt-right's fundamental beliefs have been touched upon by academics and journalists. The alt-right is defined by its white supremacist and anti-Semitic ideology, but when alt-right members talk about the women who are standing between them and their “rightful” position, their language is virtually indistinguishable from what you can find on misogynistic message boards \citep{league2018women}. The subculture is characterized by its extreme disavowal of 'political correctness' and its norms, along with the 'Social Justice Warriors' who enforce it online. They're strongly opposed to Feminist values; 'Feminism is Cancer' continues to be one of the most popular memes of the alt-right, and still serves as a rallying cry. Casting themselves as noble warriors defending the sanctity of what they love, young, white men -- railing against the enforced ‘politically correct’ critiques of the video-game industry -- engaged in coordinated, large-scale, targeted harassment while being cheered on gleefully \citep{nieborg_mainstreaming_2018}. The actions of key players set the norms for the community, and normalized a vicious version of internet ‘trolling’, giving birth to a plethora of celebrities like Milo Yiannopoulous, who modelled himself as an ‘anti-feminist, ultra-conservative bad boy’ \citep{koulouris_online_2018}. It provided a blueprint for reactionary mobilization and engagement on the internet \citep{daniels_algorithmic_2018}. 
  
In this paper, I attempt to understand how linguistic behaviour of users in online communities have changed over time, and what the relationship between the language of prominent users and the community at large looks like. I also seek to quantify how the context around key phrases has changed over time using word embedding methods. Additionally, using word embeddings, I construct dimensions of key electoral issues and track how the perception of political figures and institutions shift on these dimensions. 